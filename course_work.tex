\documentclass[bachelor, och, coursework]{SCWorks}
% параметр - тип обучения - одно из значений:
%    spec     - специальность
%    bachelor - бакалавриат (по умолчанию)
%    master   - магистратура
% параметр - форма обучения - одно из значений:
%    och   - очное (по умолчанию)
%    zaoch - заочное
% параметр - тип работы - одно из значений:
%    referat    - реферат
%    coursework - курсовая работа (по умолчанию)
%    diploma    - дипломная работа
%    pract      - отчет по практике
% параметр - включение шрифта
%    times    - включение шрифта Times New Roman (если установлен)
%               по умолчанию выключен
\usepackage{subfigure}
\usepackage{tikz,pgfplots}
\pgfplotsset{compat=1.5}
\usepackage{float}

%\usepackage{titlesec}
\setcounter{secnumdepth}{4}
%\titleformat{\paragraph}
%{\normalfont\normalsize}{\theparagraph}{1em}{}
%\titlespacing*{\paragraph}
%{35.5pt}{3.25ex plus 1ex minus .2ex}{1.5ex plus .2ex}

\titleformat{\paragraph}[block]
{\hspace{1.25cm}\normalfont}
{\theparagraph}{1ex}{}
\titlespacing{\paragraph}
{0cm}{2ex plus 1ex minus .2ex}{.4ex plus.2ex}

% --------------------------------------------------------------------------%


\usepackage[T2A]{fontenc}
\usepackage[utf8]{inputenc}
\usepackage{graphicx}
\graphicspath{ {./images/} }
\usepackage{tempora}

\usepackage[sort,compress]{cite}
\usepackage{amsmath}
\usepackage{amssymb}
\usepackage{amsthm}
\usepackage{fancyvrb}
\usepackage{listings}
\usepackage{listingsutf8}
\usepackage{longtable}
\usepackage{array}
\usepackage[english,russian]{babel}

% \usepackage[colorlinks=true]{hyperref}
\usepackage{url}

\usepackage{underscore}
\usepackage{setspace}
\usepackage{indentfirst} 
\usepackage{mathtools}
\usepackage{amsfonts}
\usepackage{enumitem}
\usepackage{tikz}
\usepackage{minted}

\newcommand{\eqdef}{\stackrel {\rm def}{=}}
\newcommand{\specialcell}[2][c]{%
\begin{tabular}[#1]{@{}c@{}}#2\end{tabular}}

\renewcommand\theFancyVerbLine{\small\arabic{FancyVerbLine}}

\newtheorem{lem}{Лемма}

\begin{document}

% Кафедра (в родительном падеже)
\chair{теоретических основ компьютерной безопасности и криптографии}

% Тема работы
\title{Генерация текстового описания к изображению с помощью нейронной сети}

% Курс
\course{3}

% Группа
\group{331}

% Факультет (в родительном падеже) (по умолчанию "факультета КНиИТ")
\department{факультета КНиИТ}

% Специальность/направление код - наименование
%\napravlenie{09.03.04 "--- Программная инженерия}
%\napravlenie{010500 "--- Математическое обеспечение и администрирование информационных систем}
%\napravlenie{230100 "--- Информатика и вычислительная техника}
%\napravlenie{231000 "--- Программная инженерия}
\napravlenie{100501 "--- Компьютерная безопасность}

% Для студентки. Для работы студента следующая команда не нужна.
% \studenttitle{Студентки}

% Фамилия, имя, отчество в родительном падеже
\author{Улитина Ивана Владимировича}

% Заведующий кафедрой
\chtitle{} % степень, звание
\chname{Абросимов М. Б.}

%Научный руководитель (для реферата преподаватель проверяющий работу)
\satitle{доцент} %должность, степень, звание
\saname{Слеповичев И. И.}

% Руководитель практики от организации (только для практики,
% для остальных типов работ не используется)
% \patitle{к.ф.-м.н.}
% \paname{С.~В.~Миронов}

% Семестр (только для практики, для остальных
% типов работ не используется)
%\term{8}

% Наименование практики (только для практики, для остальных
% типов работ не используется)
%\practtype{преддипломная}

% Продолжительность практики (количество недель) (только для практики,
% для остальных типов работ не используется)
%\duration{4}

% Даты начала и окончания практики (только для практики, для остальных
% типов работ не используется)
%\practStart{30.04.2019}
%\practFinish{27.05.2019}

% Год выполнения отчета
\date{2022}

\maketitle

% Включение нумерации рисунков, формул и таблиц по разделам
% (по умолчанию - нумерация сквозная)
% (допускается оба вида нумерации)
% \secNumbering

%-------------------------------------------------------------------------------------------

\tableofcontents

\intro

    Здесь будет введение.

\defabbr

    Здесь будет список определений, обозначений и сокращений

    % Перед анализом математической составляющей нейронной сети, работающей с конкретной задачей (в данном случае "--- с анализом изображений на предмет обнаружения объектов) стоит ввести ряд терминов и определений, которые являются фундаментом понимания работы нейросетей.

    % \textit{Граф} "--- абстрактное математическое понятие, определяющее объект, который состоит из совокупности вершин и рёбер, которые соединяют вершины.

    % \textit{Сетевая топология} "--- вид графа, вершинами которого являются конечные узлы сети, а ребрами - связи между вершинами, содержащие информацию.

    % \textit{Искусственный интеллект} (Artificial Intelligence) "--- технология создания алгоритмов, лежащих в основе проектирования интеллектуальных машин и программ, способных имитировать деятельность человека.

    % \textit{Нейронная сеть (нейросеть)} (Neural Network) "--- математическая модель, чаще всего имеющая программную интерпретацию, сутью которой является реализация деятельности, похожей на деятельность биологических нейронных сетей. Нейронная сеть используется при создании какого-либо из алгоритмов искусственного интеллекта и состоит из совокупности нейронов, соединенных между собой связями. 

    % \textit{Искусственный нейрон} "--- преобразователь одного или нескольких входных элементов в по крайней мере один выходной элемент в некоторый дискретный момент времени с определенным шагом.

    % \textit{Входной, или видимый слой} (Input layer) "--- один нейрон или совокупность нейронов в нейросети, содержащая открытые, входные и неизмененные данные, которые включают в себя доступные для изучения переменные.

    % \textit{Скрытый слой} (Hidden layer) "--- один нейрон или совокупность нейронов в нейросети, которая осуществляет вычисления и генерирует данные, не доступные для анализа и не являющиеся изначальными или конечными.

    % \textit{Выходной слой} (Output layer) "--- один нейрон или совокупность нейронов в нейросети, представляющая собой результат работы нейронной сети.

    % \textit{Признак} "--- каждый отдельный элемент информации, включаемый в представление о каком-либо анализируемом объекте.

    % \textit{Машинное обучение} (Machine Learning) "--- область науки об искусственном интеллекте, которая изучает способы создания алгоритмов, которые могут обучаться (развиваться).

    % \textit{Обучение представлений} "--- использование методов машинного обучения для определения представления.

    % \textit{Автокодировщик} "--- совокупность функции кодирования (которая преобразует входные данные в удобное для решения задачи представление) и функции декодирования, являющейся обратной по смыслу к предыдущей функции.

    % \textit{Фактор вариативности} "--- в контексте машинного обучения это концепция, помогающая получить смысл из данных той характеристики об изучаемом объекте, которая может иметь большое количество различных значений, то есть обладающая высокой вариативностью.

    % \textit{Глубокое обучение} (Deep Learning) "--- частный случай машинного обучения, который представляет из себя методы машинного обучения, основанные на обучении представлений. Осуществляет получение представлений путем их выражения через более простые представления, а формирование последних, в свою очередь, реализуется через ещё более простые представления, и так далее.

    % \textit{Компьютерное зрение} (Computer Vision) "--- область науки об искусственном интеллекте, использующая методы машинного и глубокого обучения для решения задач распознавания, классификации, мониторинга с помощью получения необходимой информации из изображения.

\section{Теоретическая часть}

    \subsection{Нейронная сеть CNN}
    \subsection{Нейронная сеть RNN и LSTM}
    \subsection{Метрики оценки качества обучения}
    \subsection{Функции потерь и функции активации}
    \subsection{Задачи, решаемые генерацией текстового описания к изображению с помощью нейронной сети}

\section{Практическая часть}

    \subsection{Описание инструментов и библиотек программной реализации}

    \subsection{Описание набора данных для обучения и теста}

    \subsection{Программная реализация алгоритма}

    \subsection{Приведение характеристик обучения и гиперпараметров}

    \subsection{Результаты обучения}

\conclusion

    Здесь будет заключение

\appendix

    \section{Код getloader.py}
    \inputminted[fontsize=\footnotesize]{python}{model-ver-2/getloader.py}

    \section{Код model.py}
    \inputminted[fontsize=\footnotesize]{python}{model-ver-2/model.py}

    \section{Код train.py}
    \inputminted[fontsize=\footnotesize]{python}{model-ver-2/train.py}

\begin{thebibliography}{99}
    \bibitem{neur} Короткий С., ''Нейронные сети: Основные положения'', [Электронный ресурс] : [статья] / URL: http://www.shestopaloff.ca/kyriako/Russian/Artificial_Intelligence/Some_publications/Korotky_Neuron_network_Lectures.pdf  (дата обращения 27.04.2021) Загл. с экрана. Яз. рус.
    % \bibitem{math} Короткий С., ''Нейронные сети: Алгоритм обратного распространения'', [Электронный ресурс] : [статья] / URL: http://masters.donntu.org/2009/fvti/trubarov/library/article2.htm (дата обращения 27.04.2021) Загл. с экрана. Яз. рус.
    % \bibitem{Neuron} Baestaens D. E., Van Den Bergh W. M., Wood D., ''Neural Network Solution for Trading in Financial Markets'', Pitman publishing, 1994 г., Яз. англ.
    % \bibitem{Network} Исаков С., ''Как работает сверточная нейронная сеть: архитектура, примеры, особенности'', [Электронный ресурс] : [статья] / URL: https://neurohive.io/ru/osnovy-data-science/glubokaya-svertochnaja-nejronnaja-set/ (дата обращения 27.04.2021) Загл. с экрана. Яз. рус.
    % \bibitem{Svertka} Дорогой Я., ''Архитектура обобщенных сверточных нейронных сетей'', [Электронный ресурс] : [статья] / URL: http://www.it-visnyk.kpi.ua/wp-content/uploads/2012/08/54_36.pdf (дата обращения 27.04.2021) Загл. с экрана. Яз. рус.
    % \bibitem{mathapp} Стариков А., ''Нейронные сети — математический аппарат'', [Электронный ресурс] : [сайт] / URL: https://basegroup.ru/community/articles/math (дата обращения 27.04.2021) Загл. с экрана. Яз. рус.
    % \bibitem{Gud} Гудфеллоу Я., Бенджио И., Курвилль А., ''Глубокое обучение'', г. Москва, Издательство ДМК, 2018 г., Яз. рус.
    % \bibitem{Gud2} Sutskever I., Martens J., Dahl G., and Hinton G., ''On the importance of initialization and momentum in deep learning.'', ICML, 2013 г., Яз. англ.
    % \bibitem{aero} Смирнов А. В., Иванов Е. С., ''Использование механизма сверточных нейронных сетей для поиска объектов на аэрофотоснимках'' [Электронный ресурс] : [статья] / URL: http://psta.psiras.ru/read/psta2017_4_85-99.pdf (дата обращения 10.04.2021) Загл. с экрана. Яз. рус.
    % \bibitem{automl} Cohen J., ''Deep Learning in Self-Driving Cars'', [Электронный ресурс] : [статья] / URL: https://becominghuman.ai/deep-learning-algorithms-in-self-driving-cars-14b13a895068 (дата обращения 03.05.2021) Загл. с экрана. Яз. англ.
\end{thebibliography}

\end{document}
